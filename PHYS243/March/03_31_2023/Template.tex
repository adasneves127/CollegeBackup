\documentclass[]{article}
\usepackage[utf8]{inputenc}
\usepackage[margin=0.5in]{geometry}
\usepackage[dvipsnames]{xcolor}
\usepackage{amsfonts, tikz, amsmath, amsthm, graphicx}

\def\docTitle{PHYS 243\\\large Dr. Thaya}
\def\docAuthor{Alex Dasneves}
\date{}

\title{\docTitle}
\author{\docAuthor}
\date{\today}


\begin{document}
\maketitle
Review for Exam:
\begin{enumerate}
    \item Introduction
    \item Vectors
    \item 1-D Motion
    \item 2-D Motion
    \item Force | Newton's Laws
    \item Force | Applications
    \item Work, Energy, Power
    \item Conservation of Energy
    \item Linear Momentum and Collisions
\end{enumerate}

\section*{Impulse and Collision}
\subsection*{Terms}
\begin{enumerate}
    \item[] \textbf{Momentum}
    \item[] \textbf{Impulse}
    \item[] \textbf{Collisions}
\end{enumerate}
The linear momentum of an object of mass m with a velocity v is defined as:\\
$\vec{p} = m\vec{v}$\\
Momentum is related to Kinetic Energy: $KE = \frac{p^2}{2m}$\\
In order to change the momentum of an object, a force must be applied. The time rate of change of momentm can be expressed as:\\
$\frac{\Delta\vec{p}}{\Delta t} = \frac{m\vec{v}_f - m\vec{v}_i}{\Delta t}$\\
$\frac{\Delta\vec{p}}{\Delta t} = m\frac{\vec{v}_f - \vec{v}_i}{\Delta t} = m\vec{a} = \vec{F}_{net}$\\
This is the Original Statement of Newton's secpnd law.
Also valid when the forces are not constant.\\
\\
When a single force acts on the object, there is an impulse delivered to the object.\\
When the force is constant, the impulse is defined as the force $F$, multiplied by the duration of time $\Delta t$.\\
$\vec{I} = \vec{F}\Delta{t}$\\
\\
\\
\begin{equation*}
    \frac{dp}{dt} = \vec{F}
\end{equation*}\\
\newpage
Force varies during impulse:\\
\begin{equation*}
    \int_{t_i}^{t_f}\vec{F}dt
\end{equation*}

What is the relation between Impulse and Momentum?\\
The theorem states that the impulse acting on the object is equal to the change in momentum of the object.\\
$\vec{I} = \vec{F}\Delta t = \frac{\Delta \vec{p}}{\Delta t}\Delta t = \Delta \vec{p}$\\
\vspace{2in}\\
In a particular crash test, a car of mass 1500 kg collides with a wall with a velocity of 15 m/s and bounce back with a velocity 2.6m/s after the collision. If the collision lasts 0.150s, find the impulse caused by the collision and the average net force exerted on the car.\\
$I = F \Delta T$
\newpage
\subsection*{Collisions}
We use the term collision to represent an event during which two particles come close to each other and interact by means of forces.\\
\begin{enumerate}
    \item The time interval during which the velocity changes from its initial to final values is assumed to be short.
    \item 
\end{enumerate}
Two particles which interact obey Newton's third law.\\
$F_{21} + F{12} = 0$\\
$m_1a_1 + m_2a_2 = 0$\\
$m_1 \frac{dv_1}{dt} + m_2 \frac{dv_2}{dt} = 0$\\
$\frac{d(m_1v_1)}{dt} + \frac{d(m_2v_2)}{dt} = 0$\\
$\frac{dp_1}{dt} + \frac{dp_2}{dt} = 0$\\
$\frac{d}{dt}(m_1v_1+m_2v_2) = 0$\\
Conservation of Linear Momentum: Momentum is conserved during a collision.\\
In an isolated system when a collision occurs, the momentum of the system is conserved.\\
\begin{enumerate}
    \item[-] An isolated system will have no external forces.
\end{enumerate}

This is known as the principle of conservation of momentum.\\
In a simple form:\\
The total momentum before a collision is equal to the total momentum after the collision.


\begin{equation*}
    m_{1}\vec{v}_{1i} + m_{2}+\vec{v}_{2i} = m_{1}\vec{v}_{1f } + m_{2}+\vec{v}_{2f}
\end{equation*}
\end{document}
