\documentclass[]{article}
\usepackage[utf8]{inputenc}
\usepackage[margin=0.5in]{geometry}
\usepackage[dvipsnames]{xcolor}
\usepackage{amsfonts, tikz, amsmath, amsthm, graphicx}

\def\docTitle{PHYS 243}
\def\docAuthor{Alex Dasneves}
\date{}

\title{\docTitle}
\author{\docAuthor}
\date{\today}


\begin{document}
\maketitle
\section*{Review}
\begin{equation*}
    \Sigma F = ma
\end{equation*}
\begin{equation*}
    F_k=\mu_k N
\end{equation*}
\begin{equation*}
    F_s=\mu_s N
\end{equation*}

$\Sigma F=ma$ toward the center:\\
$F_c=m\cdot a_c$\\
$A_c=\frac{v^2}{r}$

\section*{Chapter 7, Work, Energy, and Power}
\textbf{Power} is how fast you create energy\\

Power, worl, and energy can be constructive, or destructive.

\subsection*{Work done by a constant force}
$W = \vec{F} \cdot \Delta \vec{x}$

The work done by a constant force is given by this formula.\\This equation applies when the force is in the same direction as the displacement.\\
Units: Newton * meter = Joule\\
$J = N m = kg \cdot \frac{m^2}{s^2}$

\subsection*{Work done by a Force acting at an Angle}
$W = (F \cos \theta)\Delta x$\\
Only the component in the direction of the displacement matters.


Work can be positive or negative.\\
Positive if the force and the displacement are in the same direction.\\
Negative if the force and the displacement are in the opposite direction.\\

\subsection*{Example:}
A man cleaning a floor pulls a vacuum cleaner with a force of magnitude $F = 50.0N$ at an angle of $30.0^{\circ}$ with the horizontal. Calculate the work done by the force on the vacuum cleaner as the head is displaced $3.00m$ to the right.\\
$W = (50.0 \cdot \cos 30.0^{\circ}) \cdot 3.00$\\
$W = (\frac{\sqrt{3}}{2}) \cdot 3.00$\\
$W = 130.J$

\subsection*{Quick Method:}
The work can be represented by the dot product betweein the Force vector, and the displacement vector.\\
$W = \vec{F} \cdot \Delta \vec{X} \equiv \vec{F} \Delta \vec{x} \cos\theta$\\

\subsection*{Example: Dot Product}
$\vec{A} = 2\hat{i} + 3\hat{j}$\\
$\vec{B} = -\hat{i} + 2\hat{j}$\\
$A\cdot B = \|A\|\|B\|\cos\theta$\\
$A\cdot B = -2 + 6 = 4$\\
$\|A\| = \sqrt{13}$\\
$\|B\| = \sqrt{5}$\\
$\frac{4 = 8.06 \cos\theta}{8.06}$\\
$\frac{4}{8.06} = \cos\theta$\\
$\arccos\frac{4}{8.06} = \theta$\\
$\theta = 60.2$


If we are pulling an object with a varying force, then the work done is the integration of that curve, with resepect to time.
\large\\$W \approx \Sigma_{x_i}^{x_f} F_x\Delta x$\\
--OR--\\
$W = \int_{x_i}^{x_f}F_xdx$

$5 * 4 + \frac{1}{2}5 * 2 = 20 + 5 = 25J$

\section*{Kinetic Energy}
$K.E. = \frac{1}{2}mv^2$\\
Kinetic Energy can be used to do work.\\
Ex. The moving hammer has kinetic energy and can do work on the nail.

\subsection*{Work-Kinetic Energy Theorem}
When work is done by a net force on an object, and the only change is it's speed, the work done is equal to the change in the object's kinetic energy.\\
$W_{net} = KE_f-KE_i = \Delta KE$

\textbf{Problem:}\\
A $6.0 kg$ block initially at rest is pulled to the right along a frictionless, horizontal survace by a constant forizontal force of $12N$. Find the block's speed after it has been moved $3.0m$\\
$W = F\Delta x \equiv W = 12 * 3 = 36J$\\
$KE = \frac{1}{2}mv^2 \equiv 36J = \frac{1}{2}mv^2 = \frac{1}{2}6.0kg * v^2 = 3.0 v^2 $\\
$\frac{36}{3} = v^2 \equiv \sqrt{12} = \sqrt{v^2} \equiv 3.46 = v \approx v = 3.5$


\subsection*{Power}
Average:
$\bar{P} = \frac{\Delta W}{\Delta p} $\\
Instant:
$P = \frac{dW}{dT}$\\
$W = \frac{J}{s} = \frac{km * m^2}{s^3}$\\
$hp = 746W$\\

Units of Energy defined in terms of units of power:
kWh -- Kilowatt hour -- Unit often used on power bills


\subsection*{Potential Energy}
Potential Energy is associated with the position of the object within some system.\\
Examples:\\
\begin{itemize}
    \item Gravitational
    \item Spring
    \item Magnetic
    \item Electrical
\end{itemize}

The enery associated with the relative position of an object in space near the Earth's surface. If an object of mass $m$ is at a height $y$, then the GPE is given as:\\
$PE = mgy$\\
Work and Potential Energy:\\
Evaluating the difference of the PE at any 2 points in a object's path gives the negative work done by the gravitational force between those 2 points.\\
\\
\textbf{Conservative Forces}: Work and enery associated with the force can be recovered\\
Ex. Gravity, Springe Force, Electromagnetic Forces\\
\textbf{Non-conservative Forces}: The forces are generally dissipative and work done against it cannot easily be recovered.\\
Ex. Friction force, Air drag, Propulsive forces\\

A force is conservative if the work it does on an object moving between 2 points is independent of the path the objects take between the points. Any conserv\dots


\subsection*{Relationship between Conservative Forces and Potential Energy}
\begin{itemize}
    \item For conservative forces, the work is independent of the path
    \item For such a system, we can define a potential energy function $U$
    \begin{itemize}
        \item The work done within a system by the conservative force equals the negative change in the potential energy of a system\\
        \begin{equation*}
            W_{int} = \int_{x_i}^{x_f}F_xdx = -\Delta{U}
        \end{equation*}
        \begin{equation*}
            \Delta U = U_f-U_i = -\int_{x_i}^{x_f}F_xdx
        \end{equation*}
        \item The conservative force is related to the potential energy function through\\
        \begin{equation*}
            F_x = -\frac{dU}{dx}
        \end{equation*}
    \end{itemize}
\end{itemize}

\subsection*{Work done by Non-conservative forces}
When non-conservative forces are present, the total mechanical energy of the system is \textit(not) constant.\\
\\
The work done by all non-conservative forces acting on parts of a system equals the change in the mechanical energy of the system\\
\begin{equation*}
    W_{nc} = (KE_f-KE_i)+(PE_f-PE_i)
\end{equation*}

\subsection*{Elastic Potential Energy}
Elastic Potential Energy is related to the work required to compress a spring from its equilibrim, position to some final, arbitrary, position x
\begin{equation*}
    PE_s = \frac{1}{2}kx^2
\end{equation*}
Where $k$ is the spring constant.
\begin{enumerate}
    \item The spring is in equilibrium, neither stretched nor compressed.
    \item The spring is compressed, storing potential Energy
    \item The block is released and the postential energy is transformed to kinetic energy of the block
\end{enumerate}

\begin{equation*}
    W_{nc} = (KE_f-KE_i)+(PE_f-PE_i)+(PE_{sf}-PE_{si})
\end{equation*}
\begin{equation*}
    F_s = -\frac{dU}{dx} = -kx
\end{equation*}
This is called Hooke's Law\\
\\
\textbf{Example}\\
A spring is hung vertically and an object of mass $m$ is attached to it's lower end. Under the action of the 'load' $mg$, the spring stretches a distance $d$ from its equilibrium position.\\

\begin{enumerate}
    \item If a spring is stretched by 2.0 cm by a suspended ofbect having mass of 0.55 kg, what is the force constant of the spring?
\end{enumerate}

\begin{equation*}
    \vec{F_s} + mg = 0 \rightarrow F_s - mg = 0 \rightarrow F_s = mg
\end{equation*}
\begin{equation*}
    f = \frac{mg}{d} = \frac{(0.55)(9.80)}{2.0 \times 20^{-2}} \equiv 2.7 \times 10^2
\end{equation*}

How much work is done over the distance we stretch by?
\begin{equation*}
    W = \frac{1}{2}kd^2 \equiv -5.4 \times 10^{-2}J
\end{equation*}

\subsection*{Conservation of Energy}
We can neither create nor destroy energy, only transfered from one form or another.\\
Another way of saying; Energy is conserved.\\

\subsection*{Conservation of Mechanical Energy}
In any isolated system of objects interacting only through conservative forces, the total mechanical energy of the system remains constant.\\
\begin{equation*}
    KE_i + PE_i = KE_f + PE_f
\end{equation*}

A 3.00 kg crate slides down a ramp. The ramp is 1.00m in length and inclined at an angle of 30.0 deg. The crate starts from rest at the top, experiences a constant friction force of magitude 5.00N and continues to move a short distance on the horizontal floor after it leaves the ramp.\\
\begin{enumerate}
    \item Use energy methods to determine the speed of the crate at the bottom of the ramp.
    \item How far does the crate slive on the horizontal floor if it continues to experience a friction force of magnitude 5.00N?
\end{enumerate}

\end{document}
