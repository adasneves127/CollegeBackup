\documentclass[]{article}
\usepackage[utf8]{inputenc}
\usepackage[margin=1in]{geometry}
\usepackage[dvipsnames]{xcolor}
\usepackage{amsfonts, tikz, amsmath, amsthm, graphicx}

\def\docTitle{PHYS 243 \\\large 7.1 Worksheet: Work and the Dot Product}
\def\docAuthor{Alex Dasneves}
\date{}

\title{\docTitle}
\author{\docAuthor}
\date{\today}


\begin{document}
\maketitle
\underline{Question 1}: You walk into the gym and go over to the free weights. Someoene has left them lying on the floor. You clean the weights up and put them on the rack, a few feet above the floor. Did you do positive, negative, or zero net work.\\
In this example, we did \textbf{positive} work.\\
\\
\underline{Question 2}: You pull a box at an angle of $20^{\circ}$ above the horizontal across a level floor. The box has a mass of $4kg$. There is a coefficient of kinetic friction equal to $0.15$. After you drac the box $3.0m$ what work was done by the forec you exerted, the force of friction, the force or gravity, and the normal force? What was the total work done?\\
\\
If we assume that we are on Earth, then we can assume that $F_g = 10m/s^2$. Therefore, the normal force of the box would be equal to $4kg * 10m/s^2 = 40N$.\\\\
\begin{tabular}{|c|c|}
    variable & value\\
    $\mu$ & $0.15$\\
    $\Delta x$ & $3.0m$\\
    $F_n$ & $40N$\\
    
\end{tabular}\\
The work done by friction is -18J.\\
The work done by the force we exerted was 11J.\\
In this question, the normal force, and the force due to gravity are perpendicular to the direction of displacement, and therefore no work was done by them.


\end{document}
