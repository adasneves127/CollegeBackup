\documentclass[]{article}
\usepackage[utf8]{inputenc}
\usepackage[margin=0.5in]{geometry}
\usepackage[dvipsnames]{xcolor}
\usepackage{amsfonts, tikz, amsmath, amsthm, graphicx}

\def\docTitle{PHYS 243}
\def\docAuthor{Alex Dasneves}
\date{}

\title{\docTitle}
\author{\docAuthor}
\date{\today}


\begin{document}

\maketitle

\section*{Recap}
$W = \vec{F}\cdot d\vec{r} = F\Delta r \cos \theta$\\
$KE = \frac{1}{2}mv^2$\\
$W = \Delta KE$\\
\\
Hookes Law:
$PE_s = \frac{1}{2}kx^2$\\
$W \rightarrow -\frac{du}{dx} = Fx$\\

\section*{Worksheet 8.1}
\underline{Question 1:} You have a box of mass M. You lift the box a distance `H' and then lower it to the same position. What work does the \underline{gravity force} do on the box on the way up, on the way down, and in total? Is gravity a conservative force? If so, what is the potential energy function for gravity near the surface of the earth?\\
Since we know that work done is force over distance, and we know that gravity does negative work when it an object is being lifted, we know that the gravity did negative work on the way up, and positive work on the way down.\\
$PE_g = mgh$ implies that the Potential Energy due to gravity is equal to the mass, times the height, times the force of gravity. At the top of the travel for the box, we are at some height `h', and since we know that the weight of the box is `M', we can write the $PE_g$ as:\\
$PE_g = M \cdot h \cdot g$. If we assume $g = 10$, then it is: $PE_g = 10(Mh)$.\\
Therefore, the gravity force applies $-10Mh$ Joules on the way up, and $10Mh$ joules on the way down. This has a net of $0 J$
\\
\underline{Question 2:}\\
$W_{S_{stretch}} = -kx^2$\\
$W_{S_{relax}} = kx^2$\\
\\
\underline{Question 3:}\\
The work done by friction would be $W = \mu F \Delta r \equiv mg d \cos \theta$\\
With friction, the force is always opposite the direction of movement.\\
Since the distance, mass, and force, and direction of the friction force are all the same, we know that the work done by friction is: $2 (\mu F \Delta x)$, rather than $0$.\\
Therefore, friction is a non-conservative force.


\end{document}
