\documentclass[]{article}
\usepackage[utf8]{inputenc}
\usepackage[margin=0.5in]{geometry}
\usepackage[dvipsnames]{xcolor}
\usepackage{amsfonts, tikz, amsmath, amsthm, graphicx}

\def\docTitle{PHYS 243}
\def\docAuthor{Alex Dasneves}
\date{}

\title{\docTitle}
\author{\docAuthor}
\date{\today}


\begin{document}
\maketitle
\section*{Energy Conservation}

If there are no frictional or non-conservative forces, then the energy at the start and end of a interval should be the same.\\

\section*{Worksheet 8.2}
\underline{Problem 1} A roller coaster cart has a mass of 350kg. When the cart starts at the top of the hill at the point marked A, it is just barely moving. (You can assume it has no Kinetic Energy at A.) Assu,e the height of A is 100m, the height of B is 50m, and C is 10m. How fast is the cart moving at B and C?\\
$PE_{g_a} = mgh \equiv PE_g = 350 * 10 * 100 = 350\ 000J$\\
$KE_a = mv^2 \equiv 350 * 0 = 0$\\
$E_{total} = PE_{g_a} + KE_a = 350\ 000J$
\\
$PE_{g_b} = mgh \equiv 350 * 10 * 50 = 175\ 000J$\\
$KE_a = /frac{1}{2}mv^2 \equiv 175\ 000 = 175\ v^2 \equiv 1000 = v^2 \equiv 31.6 = v$\\
$E_{total} = PE_{g_b} + KE_a = 350\ 000J \equiv KE_a = 350\ 000 - PE_{g_b} = 350\ 000 - 175\ 000 = 175000$\\
\\
$PE_{g_c} = mgh \equiv 350 * 10 * 10 = 35\ 000J$\\
$E_{total} = PE_{g_c} + KE_a = 350\ 000J \equiv KE_a = 350\ 000 - PE_{g_c} = 350\ 000 - 35\ 000 = 315000$\\
$KE_a = mv^2 \equiv 315\ 000 = \frac{1}{2}350\ v^2 \equiv 1800 = v^2 \equiv 42.4 = v$\\
\\
\\
At B, the cart would be traveling at \(31.6 m/s\), and at C, the cart would be moving at $42.4 m/s$\\
\\
\underline{Problem 2} A spring has a constant of 180 N/m. The spring is held horizontally, parallel to a frictionless surface. A 0.30 kg box is placed against the spring, and the spring is compressed 0.40m. After the box is released by the spring, how fast will it be moving?

\end{document}
