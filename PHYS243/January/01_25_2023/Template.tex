\documentclass[]{article}
\usepackage[utf8]{inputenc}
\usepackage[margin=0.5in]{geometry}
\usepackage[dvipsnames]{xcolor}
\usepackage{amsfonts, tikz, amsmath, amsthm, graphicx}

\def\docTitle{PHYS 243}
\def\docAuthor{Alex Dasneves}
\date{Wednesday, January 25\textsuperscript{th}, 2023}

\title{\docTitle}
\author{\docAuthor}
\date{\today}


\begin{document}
\maketitle
\section*{Vector Components}
\begin{equation*}
    \overrightarrow{A}_x = \overrightarrow{A}\cos\theta
\end{equation*}

\begin{equation*}
    \overrightarrow{A}_y = \overrightarrow{A}\sin\theta
\end{equation*}
When we draw components in Physics, we draw them as a vector, with dashed lines.\\
Cosine is always on the side with the given $\theta$ angle. $\rightarrow V_x$\\
Components are always perpendicular to each other.\\
Components are always less than $\lVert \overrightarrow{v} \rVert$.\\
\begin{equation*}
    \overrightarrow{A} = \overrightarrow{A}_x + \overrightarrow{A}_y
\end{equation*}
\begin{equation*}
    \tan\theta = \frac{\overrightarrow{A}_y}{\overrightarrow{A}_x}
\end{equation*}

If $F = 75$, and $\theta = 30^{\circ}$

$F_x = 75 * \cos(30) = 64.95N \equiv 65N$\\

$F_y = 75 * \sin(30) = 37.5N \equiv 38$\\

Only adjust sigfigs at the end of calculations. 

\section*{Vector Multiplication}
\subsection*{Dot Product (Scalar Multiplication)}
\begin{equation*}
    \overrightarrow{A} \cdot \overrightarrow{B} = \lVert \overrightarrow{A} \rVert \lVert \overrightarrow{B} \rVert \cos\theta
\end{equation*}

\subsection*{Unit Vector}
3 Unit vectors in 3d Space are as follows:
\begin{itemize}
    \item[x] $\hat{i}$
    \item[y] $\hat{j}$
    \item[z] $\hat{k}$ 
\end{itemize}
If we dot multiply a unit vector by itself, we get 1.\\
If we dot multiply a unit vector by a different unit vector, we get 0.\\

Unit vectors can be written in matrix form as follows:\\
$\hat{i}$: $
    \begin{bmatrix}
        1 \\
        0 \\
        0
    \end{bmatrix}
$\\
$\hat{j}$: $
    \begin{bmatrix}
        0 \\
        1 \\
        0
    \end{bmatrix}
$\\
$\hat{k}$: $
    \begin{bmatrix}
        0 \\
        0 \\
        1
    \end{bmatrix}
$\\

Vectors $a$ and $b$ can be written as:
$\overrightarrow{A} = A_x\hat{i}+A_y\hat{j}+A_z\hat{k}$
$\overrightarrow{b} = B_x\hat{i}+B_y\hat{j}+B_z\hat{k}$
Multiplying these two vectors gives us:
\begin{equation*}
    \overrightarrow{A} \cdot \overrightarrow{B} = (A_x\hat{i}+A_y\hat{j}+A_z\hat{k}) \cdot (B_x\hat{i}+B_y\hat{j}+B_z\hat{k})
\end{equation*}
$=A_x\hat{i} \cdot (B_x\hat{i}+B_y\hat{j}+B_z\hat{k}) + A_y\hat{j} \cdot (B_x\hat{i}+B_y\hat{j}+B_z\hat{k}) + A_z\hat{k} \cdot (B_x\hat{i}+B_y\hat{j}+B_z\hat{k})$\\
$=A_xB_x + A_yB_y + A_zB_z$\\

\end{document}
