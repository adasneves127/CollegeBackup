\documentclass[]{article}
\usepackage[utf8]{inputenc}
\usepackage[margin=0.5in]{geometry}
\usepackage[dvipsnames]{xcolor}
\usepackage{amsfonts, tikz, amsmath, amsthm, graphicx}

\def\docTitle{PHYS 243}
\def\docAuthor{Alex Dasneves}
\date{Monday, January 30\textsuperscript(th), 2023}

\title{\docTitle}
\author{\docAuthor}

\begin{document}
\maketitle

\section{Introduction}
\begin{enumerate}
	\item Physical Quantities and unit dimensions
	\item Measurements with uncertainty
	\item Add/Subtract -> Decimal
	\item Multiply/divide -> Sig Figs
	\item Avg and StDev
	\item Rough Estimations
\end{enumerate}

\section{Vectors}
\begin{enumerate}
	\item Add and Subtract
	\item Scalar Product (Dot Product)
	\item Vector Product (Cross Product)
	\item $\hat{i}, \hat{j}, \hat{k}$ unit vectors
\end{enumerate}


\section{Words to describe Moving Objects}
\begin{enumerate}
	\item Acceleration | Vector
	\item Velocity | Vector
	\item Distance | Scalar
	\item Displacement | Vector
	\item Time | Scalar
\end{enumerate}

Vectors should be annotated as $\vec{v}$

In this chapter we weill study about simple motion that occur in a straight line (Motion in one dimension)\\
Usain Bolt: Top 100m Time: 9.58s\\
Virtually Nothing goes in a staight line.\\
We will be using a simplification model that is called particle model, where objects are treated as particles.\\
We will learn about velocity, acceleration and position change during motion\\
We will learn how to draw motion diagrams and derive equations that govern the motion in one dimension.

\subsection*{Position}
During motion in one-diemension (1-D for short) the object we are describing is tracelling in a straight level line.\\
The most basic quantity to keep track of is the position (Location of the object)\\
In order to quantitatively describe its position, you need to pick an origin.\\
The choice of an origin is arbitrary -> Everyone could have different posistions for the object because they have chosen different origins.

\paragraph{Question:} What is the difference between distance and displacement?\\
Displacement: Straight line, with direction and distance as the magnitude.\\
Distance: The total distance traveled (not crow distance).

\subsection{Mini Experiment: Position Graphs}


\subsection*{Average Speed and Average Velocity}
What is the difference between Speed and Velocity?\\
Average speed is defined as the total distance traveled in a unit time. $v_{avg} \equiv \frac{d}{\Delta t}$\\
Average Velodity is defined as the total displacement taken in a unit time. $v_{x,avg} = \frac{\Delta x}{\Delta t}$

\begin{tabular}{|c|c|c|}
	Position&t(s)&x(m)\\
	\hline
	A&$0$&$30$\\
	B&$10$&$52$\\
	C&$20$&$38$\\
	D&$30$&$0$\\
	E&$40$&$-37$\\
	F&$50$&$-53$
\end{tabular}
%//TODO: Insert a number line with the data from the table above.

Sometimes it is easier to draw 2 graphs, discerning between direction.

During the total ovservation, $A\Rightarrow F$,\\
What is the car's average speed? $\frac{127}{50}$m/s, $\equiv 2.5$m/s\\
What is the car's total displacement? $\frac{83}{50}$m/s $\equiv 1.7$m/s\\
\end{document}
