\documentclass[]{article}
\usepackage[utf8]{inputenc}
\usepackage[margin=0.5in]{geometry}
\usepackage[dvipsnames]{xcolor}
\usepackage{amsfonts, tikz, amsmath, amsthm, graphicx}

\def\docTitle{PHYS 243\\\large Week 1}
\def\docAuthor{Alex Dasneves}
\pagenumbering{gobble}

\title{\docTitle}
\author{\docAuthor\\Instructor: Dr. Thaya}
%\date{January 18\textsuperscript{th}, 2023}


\begin{document}
\maketitle
\section*{Week Highlights}
\begin{itemize}
    \item Syllabus was handed out
    \item Did personal introductions
    \item Study mostly things in motion and fluid dynamics
    \item Lab period not only for lab. Lecture period not only for lectures.
            Lecture might happen on lab day, lecture might happen on lab day. "Studio Style"
    \item SLA Worksheet \textbf{OR} SLA Session | 2\% of grade.
    \item Textbook through open-stax. Free\dots
\end{itemize}

\section*{What is Physics?}
\begin{itemize}
    \item The study of motion
    \item The explanation to everything
    \item The study of the physical world
    \item \textit{The goal of Physics is to provide a quantitative understanding of certain basic phenomena that occur in our Universe.}
\end{itemize}

\section*{Realms of Physics}
\begin{tabular}{|c|c|c|}
    \hline
    &SMALL & BIG\\
    \hline
    Fast&Relativistic Quantum Mechanics&General Relativity\\
    Slow&Non-Relativistic Quantum Mechanics&Classical Physics\\
    \hline
\end{tabular}
\\
\paragraph*{Physics 1 Covers}
\begin{itemize}
\item Thermodynamics
\item MechanicsFluids
\item Oscillations and waves
\end{itemize}
\paragraph*{Physics 2 Covers}
\begin{itemize}
\item Electromagnetism
\item Optics
\end{itemize}

\paragraph*{What is the diff between qualitative and quantitative measurements?} Qualitative measurements are measurements that are not numerical (Categories). Quantitative measurements are measurements that are numerical.
Answers such as "The line is long" are meaningless, since we don't have a reference point as to what "Long" is.

Mechanics uses 3 fundamental quantities:
\begin{itemize}
    \item Length
    \item Mass
    \item Time
\end{itemize}
Speed is $\frac{L}{T}$
\\
\\
Energy is $\frac{1}{2}mv^2$
\\
\\
Density is $\frac{M}{L^3}$
For more examples, check PhysicalQuantities.pdf

\section*{Units}
\textbf{SI Units are the standard units of measurement for all sciences}
\begin{itemize}
        \item Meter | Distance | M
        \item Kilogram | Mass | Kg
        \item Seconds | Time | S
        \item Kelvin | Temperature | K
        \item Ampere | Electric Current | A
        \item mole | Amount of substance | mol
        \item Candela | Luminous intensity | cd
        \item Radian | Angle | rad 
\end{itemize}


Number Grouping:\\
Group digits by 3s, starting from the decimal point, and moving outwards.
Examples:
\begin{itemize}
    \item $100,000 \Rightarrow 100\ 000$
    \item $10,000.3251452 \Rightarrow 10\ 000.325\ 145\ 2$
\end{itemize}

\textbf{Accurate Vs. Precise}
\begin{itemize}
    \item Accurate: Close to the true value
    \item Precise: Close to each other
\end{itemize}

\end{document}
