\documentclass[]{article}
\usepackage[utf8]{inputenc}
\usepackage[margin=0.5in]{geometry}
\usepackage[dvipsnames]{xcolor}
\usepackage{amsfonts, tikz, amsmath, amsthm, graphicx}

\def\docTitle{PHYS 243}
\def\docAuthor{Alex Dasneves}
\date{Monday, January 23\textsuperscript{rd}, 2023}

\title{\docTitle}
\author{\docAuthor}
\date{\today}


\begin{document}
\maketitle

Examples:
\begin{itemize}
    \item $160.0 \pm 0.1$cm | correct
    \item $160 \pm 0.1$cm | incorrect
    \item $160$ cm | incorrect
    \item $160.0$ cm | incorrect
    \item $163.3 \pm 0.1$cm | incorrect
    \item $163.3 \pm \sigma x$ cm | correct
    \item $160.0 \pm 0.1$ | incorrect
\end{itemize}
\section*{Rounding when Adding and Subtracting}
\textbf{When adding or subtracting, round the result to the smallest number of decimal places of any term in the sum (or difference)}
\\
Ex. If someone measures a length of a rod and reports as 100cm and cuts a part of it and measured as 5.6cm, what is the length of the remaining part?
\begin{itemize}
    \item $100 - 5.6 = 94.4$cm | incorrect
    \item $100 - 5.6 = 94$ cm | correct
\end{itemize}

Say the initial length has been changed to $100.00$cm, what is the length of the remaining part?
\begin{itemize}
    \item $100.00 - 5.6 = 94.4$cm | correct
    \item $100.00 - 5.6 = 94.40$ cm | incorrect
\end{itemize}

General Rule: if $A$ has 3 decimal places and $B$ has 2 decimal places, then round to 2 decimal places when adding or subtracting, as $B$ has the smaller precision.

\section*{Estimates -- Order of magnitude Calculations}
In some cases an exact answer may be difficult or impossible.\\
Due to mathematical reasons or limited information available.\\
\textbf{APproximation based on a number of assumptions.\\May yield a good estimate}\\
Order of magnitude (Based on powers of 10) approximations are commonly used.
Estimate the number of cells un the human body.\\
If 1 fL = $1 \times 10^{-15}$L, how many m\textsuperscript{$3$} in a fL?\\
$1\times 10^{-15} \times \frac{0x001m^3}{1L} = 1*10^{-15} * \frac {1 \times 10^-3}{1L} = 1x10^{-18}m^3$\\
$1*10^{-18}m^{3} \rightarrow 1\mu m^3$\\
$1000 fL \rightarrow 1x10^{-18}m^3 x 1000$
$= 1x10^{-15}m^3 \Rightarrow$ Volume of 1 cell\\
Approximate a human:\\
Make a box with dimensions 2m x 0.5m x 0.1m\\
$2m \times 0.5m \times 0.1m \approx 0.1m^3$\\
Number of cells:
$\frac{V_{human}}{V_{cell}} \approx \frac{0.1m^3}{1*10^-15} \approx 0.1 * 10^15 \approx 10^{14}$\\
Mr. Google says: 100 trillion cells in the human body.\\

\section*{Trigonometry Review}
$\sin\theta = \frac{O}{H}$\\
$\cos\theta = \frac{A}{H}$\\
$\tan\theta = \frac{O}{A}$\\
$\cot\theta = \frac{A}{O}$\\
$\sec\theta = \frac{H}{A}$\\
$\csc\theta = \frac{H}{O}$\\
Pythagorean Theorem: $A^2 + B^2 = C^2$\\
\begin{itemize}
    \item To find an angle, you need the inverse trig function.
        For example: $\sin^{-1}(0.707) = 45^{\circ}$
\end{itemize}
Be sure that calculator is in correct mode.

\section*{Coordinates}
\subsection*{Cartesian}
\begin{itemize}
    \item $x$ is horizontal
    \item $y$ is vertical
    \item $z$ is depth
\end{itemize}
Points are labeled as $(x,y,[z])$\\
\subsection*{Polar}
\begin{itemize}
    \item $r$ is the distance from the origin
    \item $\theta$ is the angle from $\theta=0$
\end{itemize}
Points are labeled as $(r,\theta)$\\
\subsection*{Conversions}
\paragraph*{Polar to Cartesian}
\begin{itemize}
    \item $x = r\cos\theta$
    \item $y = r\sin\theta$
\end{itemize}
\paragraph*{Cartesian to Polar}
\begin{itemize}
    \item $r = \sqrt{x^2 + y^2}$
    \item $\theta = \tan^{-1}\left(\frac{y}{x}\right)$
\end{itemize}

\section*{Problem Solving Strategy}
\begin{itemize}
    \item Read the problem carefully
    \item Draw a picture
    \item Label Physical Quantities
    \item Identify principals
    \item Choose Equations
    \item Solve Equations
    \item Substitute Known Values
    \item Check Answer
\end{itemize}
\newpage
A car as it enters the highway changes its speed from 50 mph to 70 mph in 40s. What is the acceleration? (Assume the car travels in a straight line)\\

\section*{Vectors vs Scaler}
Vectors have magnitude and direction.\\
Vectors are denoted as $\overrightarrow{A}$
Scalars have magnitude only.\\

\section*{Vector Addition and Subtraction}
When they are in the same line, we can add/sub their magnitudes.

To add 2 vectors, we can use the following method:
Define 2 vectors, $\overrightarrow{AB}$ and $\overrightarrow{CD}$\\
Translate $\overrightarrow{AB}$ to the origin.\\
Translate $\overrightarrow{CD}$ to point $B$, such that $B$ is at the same position as $C$.\\
The result is the vector $\overrightarrow{AD}$\\

To subtract 2 vectors, we can use the following method:
Define 2 vectors, $\overrightarrow{AB}$ and $\overrightarrow{CD}$\\
Translate $\overrightarrow{AB}$ to the origin.\\
Translate $\overrightarrow{CD}$ to point $B$, such that $B$ is at the same position as $C$.\\
Rotate $\overrightarrow{CD}$ by $2\pi$\\
The result is the vector $\overrightarrow{AD}$\\





\end{document}
