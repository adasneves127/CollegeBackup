\documentclass[]{article}
\usepackage[utf8]{inputenc}
\usepackage[margin=0.5in]{geometry}
\usepackage[dvipsnames]{xcolor}
\usepackage{amsfonts, tikz, amsmath, amsthm, graphicx}

\def\docTitle{PHYS 243}
\def\docAuthor{Alex Dasneves}
\date{Friday, January 20\textsuperscript{th}, 2023}

\title{\docTitle}
\author{\docAuthor}
\date{\today}

\begin{document}
\maketitle
\section*{Review from Wednesday}
\subsection*{Physical Quantities}
Basic 3 Units:
\begin{itemize}
    \item Mass | M | kg
    \item Length | L | m
    \item Time | T | s
\end{itemize}

Units and Conversion

Scientific Notation:
Write each number so that the number is between 0 and 10 (Exclusive). Then, add $\times 10^n$, where n is the number of decimal points you moved.



\section*{Uncertainty in Measurements}
\begin{enumerate}
    \item No physical quantity can be determined with complete accuracy
    \item The accuracy of a measurement depends on:
        \begin{enumerate}
            \item The sensitivity of the apparatus
            \item The skill of the person carrying out the measurement
            \item Number of time the measurements carried out
        \end{enumerate}
\end{enumerate}
\subsection*{Mean and Standard Deviation}
If you make $N$ measurements of a physical quantity $A$\\
The average $\overline{A}$ is the best estimation of the quantity and the standard deviation $\sigma_{A}$ is one way of expressing the uncertainty\\
$\overline{A} = \sum_{i}\frac{A_{i}}{N}$\\
$\sigma_{A} = \sqrt{\sum_{i}\frac{(\overline{A} - A)^2}{N-1}}$

\begin{tabular}{|c|c|c|}
    80&88&96
\end{tabular}
Average: 88\\ 
St. Dev: 8


\subsection*{Measuring a whiteboard:}
\begin{tabular}{|c|c|}
    \hline
    $L_1$&$362.5 \pm 0.1$cm\\
    $L_2$&$365.5 \pm 0.1$cm\\
    $L_3$&$362 \pm 0.1$cm\\
    $L_4$&$360.5\pm 0.1$ cm\\
    $L_5$&$360.5\pm 0.1$ cm\\
    $L_6$&$360.0\pm 0.1$ cm\\
    $L_7$&$360.8\pm 0.1$ cm\\
    $L_8$&$360.0\pm 0.1$ cm\\
    $L_9$&$360.8\pm 0.1$ cm\\
    $L_10$&$362.0\pm 0.1$ cm\\
    $L_11$&$360.5\pm 0.1$ cm\\
    $L_12$&$360.5\pm 0.1$ cm\\
    $\overline{L}$&$363.\overline{3}$\\
    $\sigma L$&$1.202$\\
    \hline
\end{tabular}
\begin{tabular}{|c|c|}
    \hline
    $H_1$&$121.5 \pm 0.1$cm\\
    $H_2$&$121 \pm 0.1$cm\\
    $H_3$&$121 \pm 0.1$cm\\
    $H_4$&$121 \pm 0.1$cm\\
    $H_5$&$121 \pm 0.1$cm\\
    $H_6$&$121 \pm 0.1$cm\\
    $H_7$&$121 \pm 0.1$cm\\
    $H_8$&$121 \pm 0.1$cm\\
    $H_9$&$121 \pm 0.1$cm\\
    $H_10$&$121 \pm 0.1$cm\\
    $H_11$&$121 \pm 0.1$cm\\
    $H_12$&$121 \pm 0.1$cm\\
    $\overline{H}$&$121.1\overline{6}$\\
    $\sigma H$&$0.471$\\
    \hline
\end{tabular}

\subsection*{Significant Figures}
When multiplying or dividing, keep the amount of significant figures the same as the number with the smallest amount of digits.

\end{document}
