\documentclass[]{article}
\usepackage[utf8]{inputenc}
\usepackage[margin=0.5in]{geometry}
\usepackage[dvipsnames]{xcolor}
\usepackage{amsfonts, tikz, amsmath, amsthm, graphicx}

\def\docTitle{PHYS 243}
\def\docAuthor{Alex Dasneves}
\date{}

\title{\docTitle}
\author{\docAuthor}
\date{\today}


\begin{document}
\maketitle
\section*{Instantaneous Velocity vs Average Velocity}

\begin{equation*}
    \lim_{\Delta t \to 0} \frac{\Delta x}{\Delta t}  
\end{equation*}

\begin{tabular}{c|c}
    $t$&$x=-4t+2t^2$\\\hline
    $0$ & $0$\\
    $1$&$-4(1)+2(1^2) = -4+2 = -2$\\
    $2$ & $0$\\
    $3$ & $6$
\end{tabular}\\
\includegraphics*{Intantaneous1.png}

Average Speed depends on when we measure from and to.\\
Instantaneous does not. It only relies on the velocity at the exact moment.\\
We can do this by taking a derivative.

\section*{Acceleration}
\begin{equation*}
    a_{avg} \equiv \frac{\Delta v_x}{\Delta t} = \frac{v_f-v_i}{t_f-t_i}
\end{equation*}

\begin{equation*}
    \vec{a_x} = \lim_{\Delta t \to 0} = \frac{\Delta V_x}{\Delta t} = \frac{dv_x}{dt} \equiv \frac{d}{dt}(\frac{dx}{dt})
\end{equation*}
\begin{equation*}
    a_x = \frac{d^2x}{dt^2}
\end{equation*}

\newpage
An object is only slowing down when it's acceleration is opposite it's velocity.\\
An object whose acceleration is negative is speeding up if it's velocity is also negative.

\end{document}
