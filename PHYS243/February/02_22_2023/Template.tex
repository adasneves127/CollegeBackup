\documentclass[]{article}
\usepackage[utf8]{inputenc}
\usepackage[margin=0.5in]{geometry}
\usepackage[dvipsnames]{xcolor}
\usepackage{amsfonts, tikz, amsmath, amsthm, graphicx}

\def\docTitle{PHYS 243}
\def\docAuthor{Alex Dasneves}
\date{}

\title{\docTitle}
\author{\docAuthor}
\date{\today}


\begin{document}
\maketitle
\section*{Chapter 4: Force and Motion}
Force: any influence that causes or tries to cause an object to move

2 types:\
\begin{itemize}
\item Contact Forces
\item Non-contact forces
\end{itemize}

\begin{enumerate}
    \item m
    \item The acceleration of an object is direcrlt proportional to the net force acring on it and inversely proportional to its mass.\\
    Force is the cause of change in moteion, as measured by the acceleration.
    \begin{equation}
        \Sigma \vec{F} = m\vec{a}
    \end{equation}
    \item If two objects interact, the force 
    \begin{equation}
        \vec{F}_{ab} = -\vec{F}_ba
    \end{equation}
    \begin{itemize}
        \item The normal force and the force of gravity are the forces that act on a body.
        \item In a free-body diagram, you want the forces acting on a particular object.
        \item Let the object be represented by a point.
    \end{itemize}
\end{enumerate}

\textbf{Ex.}\\
A hockey puck having a mass of $0.30$kg slides on the frictionless horizontal surface. Two hockey sticks strike the puck sumutaneously, exerting forces on the puck.\\
The force $F1$ has a magnitude of $5.0$N, and the force F2 has a magnitude of $8.0$N.\\
Determine both the magnitude and direction of the pucks acceleration.
\end{document}
