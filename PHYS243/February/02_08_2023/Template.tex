\documentclass[]{article}
\usepackage[utf8]{inputenc}
\usepackage[margin=0.5in]{geometry}
\usepackage[dvipsnames]{xcolor}
\usepackage{amsfonts, tikz, amsmath, amsthm, graphicx}

\def\docTitle{PHYS 243}
\def\docAuthor{Alex Dasneves}
\date{}

\title{\docTitle}
\author{\docAuthor}
\date{\today}


\begin{document}
\maketitle

\textbf{Worksheet: 1D Problems}

\begin{enumerate}
	\item A car is moving at 12m/s when the driver steps on the breaks. The breaks apply a uniform deceleration equal to -5.2m/s. How much time does it take for the car to come to rest?\\
	If the car is moving at 12m/s, and is slowing down at 5.2m/s, then:\\
		$\frac{12m/s}{-5.2m/s^2} = 2.3$ seconds.\\
	\item Find out how far the car tracels during this time. Do this using the time you just found.\\
		If the car is decelerating at $5.2m/s^2$ for 2.3 seconds, then we can get the following equation for the car's position:\\
		$$v_f=x_0+v_0t+\frac{1}{2}at^2 \equiv v_f=0+12*2.3 + \frac{1}{2}(-5.2)(2.3)^2 = 13.84$$
	\item Compute the distance the car tracels in a different way, using only the initial and final speeds and not the time.\\
		If we cannot use the time, then we must use the 4th equation of kinematics.\\
		$$v_f^2=v_0^2+2a(\Delta x) \equiv v_f^2 = 12^2 + 2(-5.2)(\Delta x) \equiv \Delta x = \frac{12^2}{2*(5.2)} = 13.8$$
	\item Suppose now that there is a barrier 11m in front of the car. How fast is the car going when it hits the barrier assuming the same initial speed and acceleration as above?\\
		If the barrier is at 11m, then we can use the 4th equation.\\
		$$v_f^2=v_0^2+2a(\Delta x) \equiv v_f^2 = 12^2+2(-5.2)(11) = 144-114.4 = 29.6 \equiv \sqrt{v_f^2} = \sqrt{29.6} = v_f = \sqrt{29.6} = 5.44m/s$$
\end{enumerate}

\textbf{NOTE:} The 4 kinematic equations are only useable under constant acceleration.

\textbf{Example:} A jet lands on an aircraft carrier at a speed of 140 mi/hr (~ 63m/s).
Variables:\\
$v_i = 140mi/h\\v_f=0\\x_0=0$
\begin{enumerate}
	\item[a] What is it's acceleration (assumed constant) if it stops in 2.0 seconds due to an arresting cable that snags the jet and brings it to a stop?\\
		$(A\rightarrow B) \rightarrow v_f=v_0+at \equiv 0=63 + a(2.0) \equiv 63 = a(2.0) \equiv \frac{140}{2} = a \equiv -31.5m/s^2$
	\item[b] If the jet touches down at $x_i = 0$, what is it's final position?
		$$(a\rightarrow b)\rightarrow x_f=x_0+\frac{v_f+v_0}{2}t \equiv x_f = 0+\frac{0+63}{2} \cdot 2.0$$\\
		$$x_f = \frac{63}{2} \cdot 2.0 = 63$$\\
		The airplane will stop $63$ meters after it touches down.
\end{enumerate}
\newpage
If we have a car that accelerates at $5 m/s^2$ for 5 seconds, how long will it take to get to 200 m?\\
Vars:\\
$v_0 = 0\\
a = 5m/s^2$

Split this into smaller questions. 
\begin{enumerate}
	\item How fast will it take to get up to speed?\\
	At point $B$, the car will be travelling at 25m/s. It will take 5 seconds to get there.
	\item How far will it have traveled in that time?\\
	By point C, we need to know how far we have travelled.\\
	$$x_f = x_0+\frac{v_f+v_0}{2}t \equiv x_f = \frac{25 + 0}{2} 5 = 12.5 * 5 = 62.5m$$
	At the end of acceleration, we will be 62.5 meters from the start.
	\item How long will it take to travel the rest of the distance?\\
	The remaining distance is: 137.5m. We are travelling at 25m/s.
	$$x_f=x_0+v_0t+\frac{1}{2}at^2 = 200=62.5+25t+0 \equiv 137.5 = 25t \equiv 5.5s$$
\end{enumerate}
Therefore, the remaining 137.5 meters are traveled in 5.5 seconds.\\
If we add together all the times, we get: $5.5 + 5 = 10.5s$\newpage

A stone thrown from the top of a building is given an initial velocity of 20.0m/2 straight upward. The stone is launched 50.0m above the dround, and the stone just misses the edge of the roof on its way down.\\
Vars:\\
$T_A = 0$ (The time the stone leaves the throwers hand at position A).
$V_iA = 20m/s$\\
$A = -10m/s^2$\\
$V_B = 0$ (The peak of the stone's height)\\


\end{document}
