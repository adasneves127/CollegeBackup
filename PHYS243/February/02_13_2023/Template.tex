\documentclass[]{article}
\usepackage[utf8]{inputenc}
\usepackage[margin=0.5in]{geometry}
\usepackage[dvipsnames]{xcolor}
\usepackage{amsfonts, tikz, amsmath, amsthm, graphicx}

\def\docTitle{PHYS 243}
\def\docAuthor{Alex Dasneves}
\date{}

\title{\docTitle}
\author{\docAuthor}
\date{\today}


\begin{document}
\maketitle

The velocity of an object is given by $v = 5\hat{i} + 3\hat{j} t$, where the unit vectors $\hat{i} \& \hat{j}$ are used, and $t$ stands for time.
\begin{enumerate}
    \item Write a sentence that describes this object's velocity.\\
    The particle is launched from a vertical position of 0, and a horizontal position of 0.\\
    The particle is launched at an angle of 30 degrees, and with a velocity of about 6 distance units per unit of time.\\
    Velocity will be increasing, since t has a coefficient of 3.\\
    Moves with constant velocity in the x direction of 5 units.\\
    Object is moving in the positive direction with increasing vertical component.
    \item $\vec{v} = 5\hat{i} + 6\hat{j}$
    \item m
    \item $\sqrt{5^2 + 6^2} = \sqrt{61} \equiv 7.81$
    \item Since this is a problem in 2D, then the acceleration has 2 components. In the direction of $\hat{i}$, there is no acceleration. In the direction of $\hat{j}$, there is an acceleration of 3 units of distance per division of time squared.\\
    \begin{equation*}
        \vec{a} = \frac{dx}{dt}\hat{i} + \frac{dy}{dt}\hat{j} + \frac{dz}{dt}\hat{k} = 0\hat{i} + 3\hat{j} + 0\hat{k} = 3\hat{j} 
    \end{equation*}
    The acceleration is $3\hat{j}$, and it is constant. We can tell, because there is no time component in the equation.
\end{enumerate}

\end{document}
