\documentclass[]{article}
\usepackage[utf8]{inputenc}
\usepackage[margin=0.5in]{geometry}
\usepackage[dvipsnames]{xcolor}
\usepackage{amsfonts, tikz, amsmath, amsthm, graphicx}

\def\docTitle{MATH261\\\large Dr. Vignon Oussa}
\def\docAuthor{Alex Dasneves}

\title{\docTitle}
\author{\docAuthor}
\date{\today}


\begin{document}
\maketitle
\section*{Integration Rules for a Vector-Valued Function}
If we know the acceleration of a particle, can we determine the position at a given time?\\
Yes, if we integrate the acceleration, we get the velocity. Integrate again, and we get position.\\
\begin{equation*}
    r: \mathbb{R} \to \mathbb{R}^n
\end{equation*}
\begin{equation*}
    r(t) = \langle{f_1(t),f_2(t), \cdots,f_n(t)}\rangle
\end{equation*}
\begin{equation*}
    \int r(t)dt = \int\langle{f_1(t),f_2(t), \cdots,f_n(t)}\rangle
\end{equation*}

\begin{equation*}
    =\langle{\int f_1(t),\int f_2(t), \cdots,\int f_n(t)}\rangle
\end{equation*}

\textbf{Ex.}\\
$r(t) = \langle{2\cos{t},\sin{t},2t}\rangle\\
\int{r(t)} = \langle{\int 2\cos{t}dt,\int\sin{t}dt,\int 2t dt}\rangle\\
= \langle{2\sin{t} + c_1,-\cos{t}+c_2,t^2+c_3}\rangle \equiv \langle{2\sin{t},-\cos{t},t^2}\rangle + C, C\in\mathbb{R}^3$

Definite integrals work in the same method.\\

\subsection*{Arclength and Curvature}
Let $r$ be a curve, such that\\
$r(t) = \langle{f(t),g(t)}\rangle, t\in[a,b]$\\
and $f, g$ are continuous,\\

The length of the curve from $t = a$ to $t = b$ is obtained as follows:\\
\begin{equation*}
    \int_a^b\sqrt{f'(t)^2+g'(t)^2}dt
\end{equation*}
Suppose that r(t) is the position of a particle at time $t$. Then, $r'(t)$ would be the velocity of the particle at time t.\\

$\lim_n\to\inf \sum_{k=1}^{n} \|r(t_k)\|\Delta t_k \equiv \int_a^b\|r'(t)\|dt$\\


\textbf{Ex}
$r(t) = \langle{\cos(t),\sin(t)}\rangle$\\
$t \in [0, 2\pi]$\\

velocity: $r'(t)=\langle{-sin(t),cos(t)}\rangle$\\
$\|r'(t)\| = 0$\\
$L = \int_{0}^{2\pi}\|r'(t)\|dt \equiv \int_{0}^{2\pi}dt = [t]_0^{2\pi} = 2\pi$
\newpage

\textbf{Ex} $r(t) = \langle{t,t^2,t^3}\rangle, t\in[1,2]$\\
$r'(t) = \langle{1,2t,3t^2}\rangle$\\
$\|r'(t)\| = \sqrt{1 + 4t^2 + 9t^4}$\\
$\int_1^2\sqrt{1 + 4t^2 + 9t^4}dt$ ??\\
Reparametize the function as: $t = e^u$, and therefore $\ln{1} \leq \ln{t} \leq \ln(2)$\\
$0\leq{u}\leq\ln{2}$\\
Therefore, \\
$s(u) = \langle{e^u,e^{2u},e^{3u}}\rangle$\\
$s'(u) = \langle{e^u,2e^{2u},3e^3u}\rangle$\\
$\|s'(u)\| = e^u \sqrt{1 + 4e^{2u} + 9e^{3u}}$\\
$\int_{0}^{\ln{2}} \sqrt{1 + 4e^{2u} + 9e^{3u}} e^u du$\\
How do we know that the original parameterized function will be equal to the reparameterized function?\\
How do we know that 2 different parametrizations of the same curve gives the same arclength?\\



The curvature of a curve at a point is a measure of how quickly a curve changes direction at that point.\\
$\kappa=\|\frac{dT}{ds}\|$\\
$dT$ is the change in direction of the unit vector.\\
$ds = \|r'(t)\|dt$
$\kappa$ is the inverse of the magnitude of the tangent at $t$

\begin{equation}
    \int_{0}^{\infty } x^2-2x^3dx
\end{equation}
    
\end{document}