\documentclass[]{article}
\usepackage[utf8]{inputenc}
\usepackage[margin=0.5in]{geometry}
\usepackage[dvipsnames]{xcolor}
\usepackage{amsfonts, tikz, amsmath, amsthm}
\usepackage{graphicx}

\def\docTitle{MATH261\\\large Dr. Vignon Oussa}
\def\docAuthor{Alex Dasneves}

\title{\docTitle}
\author{\docAuthor}
\date{\today}


\begin{document}
\maketitle

\section*{Partial Derivatives}
Suppose we start with $f$ be a function of 2 variables.\\
Suppose we let $x$ vary while keeping $y$ fixed.\\
$y = b$ where $b$ is a constant.\\

$f(x,y)$\\
We obtain a single variable function. We can call it $g(x, b)$\\
If $g$ has a derivative at $a$, then we call it the \textbf{Partial Derivative} of $f$ with respect to x at $(a, b)$\\
This is denoted as $f_x(a,b)$
\begin{enumerate}
    \item $f_x(a,b) = f'(a)$
    \item $f_y(a,b) = h'(b)$
\end{enumerate}

$g'(a) = \lim_{h \to 0}\frac{g(a+h)-g(a)}{h}$\\
Therefore:\\
$f_x(a,b) = \lim_{h \to 0}\frac{f(a+h,b)-f(a, b)}{h}$\\
\\
If we have a plane in 3D, denoted by a function $f(x, y)$, and we make a plane which is parallel to the X, Z plane, then we get a cross section of the plane.\\
\\
The derivative of this cross section is the partial derivative we got.\\
\\
If we now do this, but with letting y vary, and x stay constant, then we get:\\
$h(y)=f(a,y)$\\

$h'(b) = \lim_{h\to 0}\frac{h(b+h)-h(b)}{h}$\\
$f_y(a,b) = \lim_{h\to 0}\frac{f(a, b+h - f(a,b))}{h}$\\
\\
If we consider a plane, parallel to the YZ plane, then we get a cross section of the curve, which is orthagonal to the original cross section.\\
The partial derivative of this function is the slope of the curve from the cross section along the YZ plane.\\

\subsection*{Rules}
$f_x(x,y)$ Treat x as a variable, and y as a constant.\\
$f_y(x,y)$ Treat y as a varialbe, and x as a constant.\\
let $f(x,y) = x^2+y^3$\\
$\frac{d}{dx}f(x,y) = \frac{d}{dx}(x^2+y^3)$\\
If y is a constant, then $y^3$ is still a constant.\\
\\
$\frac{d}{dx}(x^2) = 2x$\\
$\frac{d}{dx}(y^3) = 0$\\
$\frac{d}{dx}(x^2+y^3) = 2x$\\
\\
$\frac{d}{dy}(x^2+y^3)$\\
$\frac{d}{dy}(x^2) + \frac{d}{dy}(y^3)$\\
$ = 0 + 3y^2 = 3y^2$\\
\\
\textbf{Ex}
If $f(x,y) = x^3+x^2y^3-2y^2$, then find:\\
$f_x(2,1) and f_y(2,1)$\\
$f_x(x,a) = \frac{d}{dx}(x^3+x^2y^3-2y^2)$\\
$\frac{d}{dx}(x^3)+ \frac{d}{dx}(x^2y^3)-\frac{d}{dx}(2y^2)$\\
$\frac{d}{dx}(x^3) = 3x^2$\\
$\frac{d}{dx}(x^2y^3) = 2xy^3$\\
$-\frac{d}{dx}(2y^2) = 0$
$3x^2+2xy^3-0 \equiv 3x^2+2xy^3$\\
\\
\\
\\
$f_y(x,a) = \frac{d}{dy}(x^3+x^2y^3-2y^2)$\\
$\frac{d}{dy}(x^3)+ \frac{d}{dx}(x^2y^3)-\frac{d}{dx}(2y^2)$\\
$\frac{d}{dy}(x^3) = 0$\\
$\frac{d}{dy}(x^2y^3) = 3x^2y^2$\\
$-\frac{d}{dy}(2y^2) = -4y$\\
$3x^2x^2 - 4y$\\
\\
\\
$f_x(2,1) = 3(2)^2 + 2(2)(1)^3 \equiv 16$\\
\\
$f_y(2,1) = 3(2)^2(1)^2 - 4(1) \equiv 8$\\
\\
\\
$f(x,y,z) = e^xy\ln(x)$\\
$f_x(x,b,c) = \frac{d}{dx}(e^{bx}\ln(c)) = \frac{d}{dx}(e^{bx}) + \frac{d}{dx}(\ln(c)) = b(e^{bx}) + 0$\\
$f_y(a,y,z) = \frac{d}{dy}(e^{ay}\ln(c)) = \frac{d}{dy}(e^{ay}) + \frac{d}{dy}(\ln(c)) = a(e^{ay}) + 0$\\
$f_z(a,b,z) = \frac{d}{dz}(e^{ab}\ln(z)) = \frac{d}{dz}(e^{ab}) + \frac{d}{dz}(\ln(z)) = 0 + \frac{1}{z}$\\
\\
$f_x = \ln(z) \cdot y \cdot e^{xy}$\\
$f_y = \ln(z) \cdot x \cdot e^{xy}$\\
$f_z = e^{xy} \cdot \frac{1}{z}$\\
\\
\\
Find $f_x, f_y$\\
$f(x,y) = (\arctan{y})^x$\\
$f_x(a,y) = (\arctan(y))^a$\\
$f_y(x, b) = \arctan(b)^x$\\
\\
\\
$f_x(a,y) = (e^{x\ln(\arctan{y})}) = \ln(\arctan(y))e^{x\ln\arctan(y)}$\\
$f_y(x, b) = e^{x\ln\arctan(y)} \equiv \frac{d}{dx}x\ln\arctan(y) \cdot e^{\ln\arctan x}$\\
$\equiv x \frac{d}{dy}(\ln(\arctan y)) = \frac{x \frac{1}{1+y^2}}{\arctan y}$
\\
\\      
$\frac{d}{ds} \frac{d}{dr} e^r \sin(s^2r+r^2)$\\
$\frac{d}{ds} e^r \frac{d}{dr}  \sin(s^2r+r^2)$\\
$\frac{d}{ds} e^r \cdot 2r \cdot \cos(s^2r+r^2)$
$2re^r \frac{d}{ds} \cos(s^2r + r^2)  \equiv 2re^r -\sin(s^2r + r^2)$

\newpage
\section*{Second Order Partial Derivatives}
$f_{xx} = \frac{d}{dx}\frac{d}{dx}f$\\
$f_{xy} = \frac{d}{dy}\frac{d}{dx}f$\\
\\
$f(x,y) = x^2y + x^3y^2$\\
$f_{xx} = \frac{d}{dx}\frac{d}{dx} x^2y + x^3y^2 \equiv \frac{d}{dx}2xy+3x^2y^2 \equiv 2y+6y^2$\\
$f_{xy} = \frac{d}{dx}\frac{d}{dy} x^2y + x^3y^2 \equiv \frac{d}{dx} x^2+2x^3y \equiv 2x+6x^2y $\\
$f_{yx} = \frac{d}{dy}\frac{d}{dx} x^2y + x^3y^2 \equiv \frac{d}{dy}x^2+2x^3y \equiv 2x+6x^2y$\\
$f_{yy} = \frac{d}{dy}\frac{d}{dy} x^2y + x^3y^2 \equiv \frac{d}{dy}x^2+2x^3y \equiv 2x^3$\\
\\
\\
$\frac{d}{dx}\sinh(xy)$\\
$\sinh{x} = \frac{1}{2}(e^x-e^{-x})$




\end{document}