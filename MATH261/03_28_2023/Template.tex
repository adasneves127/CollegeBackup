\documentclass[]{article}
\usepackage[utf8]{inputenc}
\usepackage[margin=0.5in]{geometry}
\usepackage[dvipsnames]{xcolor}
\usepackage{amsfonts, tikz, amsmath, amsthm, graphicx, cleveref}
\usepackage{../commands}

\def\docTitle{MATH261\\\large Dr. Vignon Oussa}
\def\docAuthor{Alex Dasneves}

\title{\docTitle}
\author{\docAuthor}
\date{\today}


\begin{document}
\maketitle
Equation of tangent line: $y - f(a) = f'(a)(x-a) \equiv y= f'(a)(x-a) + f(a) $\\
How do we generalize this concept?\\
\\
Toy Example:\\
$z = f(x,y) = x^2+y^2$\\
$z = x^2+y^2$\\
$x = -2$\\
\\
$z = y^2 + 4$\\
$x = -2$\\
Set \(y=t\)\\
\\
$r(t) = \langle{-2,t,t^2+4}\rangle$\\
\\
\\
\\
Let $z = f(x,y)$\\
Let $P = (a, b, f(a,b))$ be a point on the graph of $f$\\
Then, an equation for the tangent plane to this graph at $P$ is given as follows:\\
$z-f(a,b) = \frac{df}{dx}(a,b)(x-a)+\frac{df}{dy}(y-a)$
% Insert all text from board photo.
\\
\\
\textbf{Ex}\\
$f(x,y) = xe^{xy}$\\
$P = (1.1, -0.1, f(1.1, 0.1))$\\
$z - f(a,b) = \frac{d}{dx}f(a,b)(x-1) + \frac{d}{dy}f(y-a)$\\
$\frac{d}{dx}f(a,b) = \frac{d}{dx}g(x)h(x) = g'(x)h(x) + h'(x)g(x) = e^{xy} + xye^{xy} \equiv e^{xy}(1 + xy)$\\
$\frac{d}{dy}f(a,b) = xye^{xy}$\\
$z - 0.2279 = $



\end{document}