\documentclass[]{article}
\usepackage[utf8]{inputenc}
\usepackage[margin=0.5in]{geometry}
\usepackage[dvipsnames]{xcolor}
\usepackage{amsfonts, tikz, amsmath, amsthm, graphicx}

\def\docTitle{MATH261\\\large Dr. Vignon Oussa}
\def\docAuthor{Alex Dasneves}

\title{\docTitle}
\author{\docAuthor}
\date{\today}
\setcounter{section}{1}


\begin{document}
\maketitle

Find the unit vector in the direction of $v = <\frac{1}{3},\frac{2}{3},\frac{-1}{3}>$
\begin{equation*}
    Ev = \frac{v}{\lVert v\rVert}v = \frac{v}{\sqrt{\frac{1}{9}+\frac{4}{9}+\frac{1}{9}}}<\frac{1}{3},\frac{2}{3},\frac{-1}{3}> = \sqrt{\frac{1}{9}+\frac{4}{9}+\frac{1}{9}}<\frac{1}{3},\frac{2}{3},\frac{-1}{3}> = \sqrt{\frac{9}{6}}<\frac{1}{3},\frac{2}{3},\frac{-1}{3}> = <\frac{\sqrt{6}}{6},\frac{\sqrt{6}}{3},-\frac{\sqrt{6}}{6}>
\end{equation*}

A weight hangs from 2 wires, Find T1, T2.
\begin{equation*}
    E_{T1} = <\cos 50^{\circ}, \sin 50^{\circ}>
\end{equation*}
\begin{equation*}
    T_1 = \|T_1\|<\cos 50^{\circ}, \sin 50^{\circ}> = <\|T_1\|\cos 50^{\circ}, \|t_1\|\sin 50^{\circ}>
\end{equation*}
\begin{equation*}
    E_{T2} = <\cos 32^{\circ}, \sin 32^{\circ}>
\end{equation*}
\begin{equation*}
    E_{T2} = \|T_2\|<\cos 32^{\circ}, \sin 32^{\circ}> = <\|T_2\|\cos 32^{\circ}, \|t_2\|\sin 32^{\circ}>
\end{equation*}

Therefore, $T_1 + T_2 = 100j = 100<0,1> = <0, 100>$\\
\begin{equation*}
    <\|T_1\|\cos 50 + \|T_2\|\cos 32, \|T_1\|\sin 50 + \|T_2\|\sin 32> = <0, 100>
\end{equation*}
\begin{equation*}
    \|T_1\|\cos 50 + \|T_2\|\cos 32 = 0
\end{equation*}
\begin{equation*}
    \|T_1\| = -\frac{\|T_2\|\cos 32}{\cos 50}
\end{equation*}
\begin{equation*}
\|T_1\|\sin 50 + \|T_2\|\sin 32 = 100
\end{equation*}
\begin{equation*}
    \|T_2\|\frac{-\cos 32\sin 50}{\cos 50} + \|T_2\|\sin 32 = 100
\end{equation*}

\begin{equation*}
\|T_2\|(-\frac{\cos 32\sin 50}{\cos 50} + \sin 32) = 100
\end{equation*}

\begin{equation*}
\|T_2\| \cdot (-\frac{\cos 32 + \sin 32\cos 50}{\cos 50})
\end{equation*}
\begin{equation*}
\|T_2\| = \frac{100\cos 50}{\sin 32\cos 50 -\cos 32\sin 50}
\end{equation*}


\section{Dot Product}
Given $v, w \in \mathbb{R}^n$
$v = <v_1, v_2, \cdots, v_n>$
$w = <w_1, w_2, \cdots, w_n>$

\begin{equation*}
    v \cdot w = v_1w_1 + v_2w_2 + \cdots + v_nw_n
\end{equation*}
So, for example, if $v = <2, 3> and w = <1, 4>$\\
Then: $v \cdot w = 2\cdot 1 + 3\cdot 4 = 2 + 12 = 14$

Example 2:\\
$v = <1, 2, 3>$ and $w = <-1, 3, 2>$\\
$v \cdot w = 1\cdot -1 + 2\cdot 3 + 3\cdot 2 = -1 + 6 + 6 = 12$

\subsection{Dot Product Properties}
\begin{enumerate}
    \item If you take the vector times itself, you get the magnitude squared.\\
\begin{equation*}
    v \cdot v = \lVert v \rVert^2
\end{equation*}
    \item $v \cdot w = w \cdot v$ (commutative)
    \item $V \cdot (U + w) = v \cdot u + v \cdot w$
    \item If $\lambda \in \mathbb{R}$, then $\lambda v \cdot w = \lambda (v \cdot w)$
    \item If $\emptyset  * v = \emptyset$, where $\emptyset$ is the zero vector.
\end{enumerate}

\subsection{Proof}
\begin{enumerate}
    \item Write $v = <v_1, v_2, \cdots, v_n>$\\
        $v \cdot v = <v_1, v_2, \cdots, v_n> * <v_1, v_2, \cdots, v_n>$\\
        $v_1^2 + v_2^2 + \cdots + v_n^2 = \sqrt{v_1^2 + v_2^2 + \cdots + v_n^2} = \|v\|^2$
    \item 
        \item $v = <v_1, \cdots, v_n>$\\
         $u = <u_1, \cdots, u_n>$\\
         $w = <w_1, \cdots, w_n>$\\
         $v \cdot (u + w) = <v_1, \cdots, v_n> * <u_1 + w_1, \cdots, u_n + w_n>$\\
         $(v_1\cdot u_1 + \cdots v_n\cdot u_n) + (v_1\cdot w_1 + \cdots + v_n\cdots w_n)$\\
         $v \cdot u + v \cdot w$
         \item 
\end{enumerate}

\subsection{Dot Product and Angles}
Let $v, w \in \mathbb{R}^n$\\
$v * w = \|v\|\|w\|\cos\theta$\\
Where $\theta$ is the smaller positive angle between v and w.\\

\subsection{Law of Cosines}
$c^2 = a^2 + b^2 - 2ab\cos\theta$\\
Angle must be opposite the side you are trying to find.\\

Apply law of cosines to a triangle:\\
$\|w-v\|^2 = \|v\|^2 + \|w\|^2-2\|v\|\|w\|\cos\theta$\\
$(w-v)(w-v) = w\cdot w - w\cdot v - v\cdot w + v\cdot v$\\
$(w-v)(w-v) = \|w\|^2 - 2w\cdot v + \|v\|^2$\\
$(w-v)(w-v) = \|w\|^2 - 2\|v\|\|w\|\cos\theta + \|v\|^2$\\
$-2v\cdot w = -2\|v\|\|w\|\cos\theta$\\
$v*w = \|v\|\|w\|\cos\theta$
\\
\textbf{Ex.} If thevector $v, w$ have lengths $4$ and $6$, and $\theta = \frac{\pi}{3}$, find $v*w$\\
$v*w = \|v\|\|w\|\cos\frac{\pi}{3}$\\
$v*w = 4\cdot 6\cdot\frac{1}{2} = 24 * \frac{1}{2} = 12$\\

Assume that v, w are non-zero vectors.\\
Then $\cos\theta = \frac{v\cdot w}{(\|v\|\|w\|)}$\\
We can say that $\cos\theta \in [0, \pi]$.\\
$\arccos(\cos\theta) = \arccos(\frac{v\cdot w}{(\|v\|\|w\|)})$\\
$\theta = \arccos(\frac{v\cdot w}{(\|v\|\|w\|)})$\\
\subsection{Why is this useful?}
\textbf{Remark}\\
If $v \cdot w = 0$, then $\theta = \frac{\pi}{2}$. Therefore, $v, w$ are orthagonal vectors\\
If $v \cdot w$ is positive, then the angle between v and w is acute.\\
If $v \cdot w$ is negative, then the angle between v and w is obtuse.\\

\subsection*{More Examples}
\begin{enumerate}
    \item[] $v = <2, 2, -1>$ and $w = <5, -3, 2>$\\
        $\theta = \arccos(\frac{v\cdot w}{(\|v\|\|w\|)})$\\
        $= \arccos(\frac{2\cdot 5 + 2\cdot -3 + -1\cdot 2}{\sqrt{2^2 + 2^2 + -1^2}\sqrt{5^2 + -3^2 + 2^2}})$\\
        $= \arccos(\frac{12 -6 + -2}{3 \sqrt{38}})$\\
        $= \arccos(\frac{4}{3 \sqrt{38}})$\\
        $= 84^{\circ}$\\
    \item[] $v = <2,2,-1>, w=<5,-4,2>$\\
    $\theta = \arccos(\frac{v\cdot w}{(\|v\|\|w\|)})$\\
    $= \arccos(\frac{10 - 6 - 2}{\sqrt{9}*\sqrt{40}}) = 0$\\
    
\end{enumerate}

\subsection{Direction Angles}
Find the unique box sitting on the axis such that the vector $\overrightarrow{v}$ sits on the diagonal of the box.\\
The angle between the x-axis and $\overrightarrow{v}$ is $\alpha$.\\
The angle between the y-axis and $\overrightarrow{v}$ is $\beta$.\\
The angle between the z-axis and $\overrightarrow{v}$ is $\gamma$.\\
$\cos\alpha = \frac{v * i}{\|v\|}$\\
$\cos\beta = \frac{v * j}{\|v\|}$\\
$\cos\gamma = \frac{v * k}{\|v\|}$\\
Therefore:\\
$\alpha = \arccos(\frac{v * i}{\|v\|})$\\
$\beta = \arccos(\frac{v * j}{\|v\|})$\\
$\gamma = \arccos(\frac{v * k}{\|v\|})$\\

If we write $v = <v_1, v_2, v_3>$, then:\\
$i = <1, 0, 0>$\\
Therefore: $v\cdot i = v_1$\\
Therefore: $v\cdot j = v_2$\\
Therefore: $v\cdot k = v_3$\\
And therefore:\\
$\alpha = \arccos(\frac{v_1}{\|v\|})$\\
$\beta = \arccos(\frac{v_2}{\|v\|})$\\
$\gamma = \arccos(\frac{v_3}{\|v\|})$\\

\textbf{Ex.} Find the direction angles of: $v = <1,2,3>$:\\
$\alpha = \arccos(\frac{1}{\sqrt{14}})$\\
$\beta = \arccos(\frac{2}{\sqrt{14}})$\\
$\gamma = \arccos(\frac{3}{\sqrt{14}})$\\

\textit{Subtle Quote: There are 2 things that can change your mood: Jerks, and music.\\}
\newpage
\subsection{Projections}
\begin{equation*}
    proj_w V = (v\cdot e_w) * e_w
\end{equation*}
Since $proj_w v + orth_w v = v$
\begin{equation*}
    orth_w v = v - proj_w v
\end{equation*}


\section*{Homework on Dot Product}

\subsection*{Solve the problem}
\begin{enumerate}
    \item Let $v = 6i + 8j$ and $w = 10i-9j$ Compute $v\cdot w$.\\
    $v = <6,8>, w=<10,-9>$\\
    $v\cdot w = v_1 \cdot w_1 + v_2 \cdot w_2 + \cdots + v_n \cdot w_n$\\
    $v \cdot w = (6 \cdot 10) + (8 \cdot -9)$\\
    $v \cdot w = (60) + (-72) = -12$
    \item $r=\left\langle6,-8,1\right\rangle, v=\left\langle2,-9,6\right\rangle, w=\left\langle-9,-1,-3\right\rangle$ Find $(r+v)\cdot w$\\
    $v \cdot (u + w) = v \cdot u + v \cdot w$, Therefore \\
    $(r+v)\cdot w = (w \cdot r) + (w \cdot u)$\\
    $(w \cdot r) = (6*-9+-8*-1+1*-3) = -49$\\
    $(w \cdot u) = (2*-9+-9*-1+6*-3) = -27$\\
    $(w \cdot r) + (w \cdot u) = -49+-27 = -76$
    
    \item Let $a = 3j + 2k$ and $b = j+k$. Find $\frac{a\cdot b}{\|a\|\|b\|}$\\
    $a\cdot b = 3*1 + 2+1 = 3+2 = 5$\\
    $\|a\| = \sqrt{3^2 + 2^2} = \sqrt{13}$\\
    $\|b\| = \sqrt{1^2 + 1^2} = \sqrt{2}$\\
    $\|a\|\|b\| = \sqrt{13}* \sqrt{2} = \sqrt{26}$\\
    $\frac{a\cdot b}{\|a\|\|b\|} = \frac{5}{\sqrt{26}} \equiv \frac{5\sqrt{26}}{26}$
    \item Find the angle between the given vectors to the nearest tenth of a degree:
        $v = 3i + 5j, w=3i+4j$\\
        $\theta = \arccos(\frac{v\cdot w}{(\|v\|\|w\|)})$\\
        $= \arccos(\frac{3*3 + 5*4}{\sqrt{3^2+5^2}*\sqrt{3^2+4^2}})$\\
        $= \arccos(\frac{3*3 + 5*4}{\sqrt{9+25}*\sqrt{9+16}})$\\
        $= \arccos(\frac{3*3 + 5*4}{\sqrt{34}*\sqrt{25}})$\\
        $= \arccos(\frac{3*3 + 5*4}{\sqrt{34}*\sqrt{25}})$\\
        $= \arccos(\frac{9 + 20}{\sqrt{870}})$\\
        $= \arccos(\frac{29}{\sqrt{870}})$\\
        $= 0.98319208025 rad$\\
    \item 
    
\end{enumerate}

\end{document}