\documentclass[]{article}
\usepackage[utf8]{inputenc}
\usepackage[margin=0.5in]{geometry}
\usepackage[dvipsnames]{xcolor}
\usepackage{amsfonts, tikz, amsmath, amsthm, graphicx}

\def\docTitle{MATH261\\\large Dr. Vignon Oussa}
\def\docAuthor{Alex Dasneves}

\title{\docTitle}
\author{\docAuthor}
\date{Thursday, February 2\textsuperscript{nd}, 2023}


\begin{document}
\maketitle
\section{Cross Product}
    \textbf{Given} 2 vectors, with components $a=<a_1, a_2, a_3>$ and $b=<b_1, b_2, b_3>$, how do we find $c$, such that c is perpendicular to a and b.\\
    In other words, we want to find a vector $c$, such that c is perpendicular to to the plane spanned by $ab$

    %//TODO: insert a 3d photo of 2 vectors, a and b, with cross product C.
    let $C=<C_1,C_2,C_3>$, Find $C_1,C_2,C_3 \in \mathbb{R}$\\
    $C\perp a \Leftrightarrow c\cdot a = 0 \Leftrightarrow c_1a_1 + c_2a_2 + c_3a_3 = 0$ \\
    $C\perp b \Leftrightarrow c\cdot b = 0 \Leftrightarrow c_1b_1 + c_2b_2 + c_3b_3 = 0$ \\
    $c_1a_1 + c_2a_2 + c_3a_3 = c_1b_1 + c_2b_2 + c_3b_3$, for unknown $C_1,C_2,C_3$\\
    Therefore: $b_1c_2a_2 - a_1c_2b_2 + b_1c_3c_3 - a_1c_3b_3=0$\\
    $\equiv c_2(b_1a_2-b_1a_2)+ c_3(b_1a_3-a_1b_3) = 0$\\
    Let $p = b_1a_2-b_1a_2$ and $q = b_1a_3-a_1b_3$\\
    Therefore, $c_2p+c_3q = 0$
    If we set $C_2 = p$ and $c_3 = -q$, then:\\
    $c_1a_1=(a_1b_3a_2-a_1b_2a_3) \equiv c_1a_1=a_1(b_3a_2-b_2a_3)$\\
    Assume $a_1\neq 0$\\
    $c_1=b_3a_2-b_2a_3$
    $c_3=-(b_1a_2-a_1b_1)$\\
    Therefore:
    \begin{equation*}
        c=\left\langle b_3a_2-b_2a_3, b_1a_3-a_1b_3, a_1b_2-a_2b_1\right\rangle 
    \end{equation*}
    \textbf{Note:}
    \begin{enumerate}
        \item The problem as stated is ill-posed.
            It cannot be fully answered (in a unique sence) without making assumptions about the orientation of $C$, and it's magnitude.\\ We are left with Infinitely Many Solutions.
        \item The calculations used to derive the answer obtained earlier are cumbersome and tedious.
        \item The components of $c$ are not easily memorized
        \item We assumed at some point that $a_1$ is not zero, and did not cover that case
    \end{enumerate}

    \subsection*{Systematic Method}
    \textbf{Define} The cross product of a and b as the unique vector of the form $ c=\left\langle b_3a_2-b_2a_3, b_1a_3-a_1b_3, a_1b_2-a_2b_1\right\rangle $\\
    \textbf{Remark} Note that unlike the dot product, the cross product of 2 vectors is a vector, and \underline{not} a number 

    \subsection*{Matrix}
    $ \begin{bmatrix}
        a_{11} & a_{12} & \cdots & a_{1n} \\
        a_{21} & a_{2n} & \cdots & a_{2n} \\
        a_{31} & a_{32} & \cdots & a_{32} \\
        \cdots & \cdots &  & \cdots\\
        a_{n1} & a_{n2} & \cdots & a_{nn}\\
        \end{bmatrix}  $

        This is an example of a square matric of order n.
        For every matrix of order N, there are $n^2$ elements
    \subsection*{Determinant of a Matrix}
    Det. of $\begin{bmatrix} a \end{bmatrix}$ is just $a$.
    
    What is the determinant of:
    $\begin{bmatrix}
        a&b\\c&d
    \end{bmatrix}
    $

    We can write the determinant as:
    $
    det(\begin{bmatrix}
        a&b\\c&d
    \end{bmatrix})
    $\\
    We can get the determinant as: Product of the diagonal - that of the antidiagonal.

    So, for the matrix above, $ad - bc$

    Det of a 3D matrix:
    $
    \begin{bmatrix}
        a_{11} & a_{12} & a_{13}\\
        a_{21} & a_{22} & a_{23}\\
        a_{31} & a_{32} & a_{33}\\
    \end{bmatrix}$\\
    Step 1: Alternate the sign of each column.\\
    Ignore the first column and the first row.\\
    We are left with : $\begin{bmatrix}
        a_{22} & a_{23}\\
a_{32} & a_{33}
    \end{bmatrix}$\\
    We can write this as: $a_{11} (a_{22}a_{33}-a_{32}a_{23})$.\\
    Now we ignore the second column.\\
    We can write this as: $-a_{12}(a_{21}a_{33}-a_{32}a_{23})$\\
    Next, delete the third, and repeat.
    We can write this as: $a_{13}(a_{21}a_{32}-a_{23}1_{32})$\\
    When we add these terms, we get: 
    \begin{equation*}
        a_{11} (a_{22}a_{33}-a_{32}a_{23})-a_{12}(a_{21}a_{33}-a_{32}a_{23})+a_{13}(a_{21}a_{32}-a_{23}1_{32})
    \end{equation*}

    The determinant tells us the area of a unit cube after a linear transformation of space.
    If we do the determinant on a 3D graph, such as:\\
    $\begin{bmatrix}
        i_x & j_x &k_x\\
        i_y & j_y & k_y\\
        i_z & j_z & k_z
    \end{bmatrix}$\\
    Then we will get the area of the shape defined by the unit cube, no matter how distoreted it is (So long as the transformations are linear)\\
    Definitions:\\The determinant is this stupid thing that we defined, and then continue inductively\\
    The determinant of a matrix is a signed volumn for the unique pallotope, which is spanned by L vectors, where vector 1 is row 1, vector 2 is row 2, and so on.\\
    Eigenvalues. A matrix is a map (or a function) that takes a vector into another vector. It is addition and multiplication preserving.
    \textbf{Definitions}:
    \begin{enumerate}
        \item Eigenvalue: A number $\lambda$ such that $A\vec{v} = \lambda \vec{v}$
        \item Eigenvector: A vector $\vec{v}$ such that $A\vec{v} = \lambda \vec{v}$
    \end{enumerate}

    $\begin{bmatrix}
        2&3&1\\1&-4&2\\4&3&1
    \end{bmatrix}$
    $2 * det\begin{bmatrix} -4&2\\3&1\end{bmatrix}-3*det\begin{bmatrix}
        1&2\\4&1
    \end{bmatrix} * 1*\begin{bmatrix}
        1*-4\\4&3
    \end{bmatrix} \equiv 2(-4-6)-3(1-8)+1(3+16)$\\
    $\equiv 2(-10)-3(-7)+19 \equiv-20+21+19 \equiv 20$

    Suppose we have 2 vectors: $a=\left\lVert<a_1,a_2,a_3>\right\rVert \\ b=\left\lVert<b_1,b_2,b_3>\right\rVert$\\
    If we define c as: $c = a\times b$, then we can write the cross product as the determinant of the following:
    \begin{equation*}
        \begin{bmatrix}
            i&j&k\\
            a_1&a_2&a_3\\
            b_1&b_2&b_3
        \end{bmatrix}
    \end{equation*}
    This determinant works out to:
    $c=\left\langle b_3a_2-b_2a_3, b_1a_3-a_1b_3, a_1b_2-a_2b_1\right\rangle$

    $\vec{v} = 8i-9j+17k$\\
    $\vec{w} = -4i-8j-7k$\\
    \begin{equation*}
        \begin{bmatrix}
            
        i &j &k\\
        8 &-9 &17\\
        -4&-8&-7
    \end{bmatrix}
    \end{equation*}

    Determinant: $i(-9*-7-17*-8) - j(8*-7 - 17 * 4) + k(8*-8-(-9)*-4) \equiv i(199) - j(12) + k(-100)$\\
    $\equiv \left\langle 199, -12, -100 \right\rangle$\\


    Given $v, w \in \mathbb{R}^3$:
    \begin{enumerate}
        \item $v\times w \perp v$
        \item $v\times w \perp w$
        \item $\left\lVert v\times w \right\rVert = \|v\|\|w\|\sin\theta$
        \item If $v, w$ are parrallel vectors, $\theta = \pi or \theta = 0$, then the corss is a zero-vector
    \end{enumerate}

    Area of a parralellogram: $\|v\|\|w\|\sin\theta$\\
    So, $v\times w$ is a vector perpendicular to both V and W, whose magnitude is the area of the unique parallelogram spanned by $v$ and $w$


    \textbf{Theorem}
    \begin{enumerate}
        \item $a\times b = -(b\times a)$ (Anticommutative)
        \item Assume $c$ is a scalar. $(c \cdot a) \times b = c\cdot(a\times b)$
        \item $a\times(b+c) = a\times b + a\times c$ (Distributing)
        \item $(b+c)\times a =ba\times a + c\times a$ 
        \item $a\cdot (a\times b) = (a\cdot b) \times b$
        \item $a\times(b\times c) = (a\cdot c) \cdot b \cdot (a\cdot b)\cdot c$
    \end{enumerate}

    $$ a\cdot b \times c \equiv a\times \det\begin{bmatrix}
        i&j&k\\
        b_a&b_2&b_3\\
        c_1&c_2&c_3
    \end{bmatrix} \equiv \left\langle a_1,a_2,a_3 \right\rangle \cdot \left\langle b_2c_3-c_2b_3,-b_1c_3+c_1b_3, b_1c_2-c_1b_2\right\rangle$$\\
    Triple Product:
    $$a \cdot (b \times c) = \det \begin{bmatrix}
        a_1&a_2&a_3\\b_1&b_2&b_3\\c_1&c_2&c_3\\
    \end{bmatrix}$$
\end{document}