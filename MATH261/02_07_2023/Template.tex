\documentclass[]{article}
\usepackage[utf8]{inputenc}
\usepackage[margin=0.5in]{geometry}
\usepackage[dvipsnames]{xcolor}
\usepackage{amsfonts, tikz, amsmath, amsthm, graphicx}

\def\docTitle{MATH261\\\large Dr. Vignon Oussa}
\def\docAuthor{Alex Dasneves}

\title{\docTitle}
\author{\docAuthor}
\date{\today}


\begin{document}
\maketitle
$|a\cdot b\times c| = $ Volume of the polytope determined by $a, b, c$

\textbf{Ex.}
$
\begin{bmatrix}
    1&4&10\\
    2&3&4\\
    1&1&1
\end{bmatrix}
$

Determinant: $1 \begin{bmatrix} 3&4\\1&1\end{bmatrix}- 4\begin{bmatrix} 2&4\\1&1\end{bmatrix} + 10\begin{bmatrix}2&3\\1&1\end{bmatrix}$

Torque is a cross product of 2 vectors\\
We consider a force $F$ acting on a rigid bodyat a point given by a position vector $r$\\
Suppose you have a vector $r$, and a force $F$, applied at angle $\theta$. The torque relative to point $O$ is defined as:\\
$t = r \times F$\\
The torque measures the tendancy of the body to rotate about the origin. The direction of the torque indiates the axis of rotation.\\
$\|t\| = \|r\|\|F\|\sin\theta$

\textbf{Ex} A bolt is tightened by applying a 40N force to a 0.25m wrench, applied at an angle of $75^{\circ}$.\\
Find the magnitude of the torque on the center of the bolt.\\
$\|\tau\|=\|r\|\|F\|\sin{75^{\circ}}$
$= 9.66 N.m$


\section{\underline{Equation of Lines and Planes}}
A line in the xy-plane is determined by a point of the line and a direction.\\
\hspace*{0.5in} Built into the direction is the slope.\\
$m = \frac{\Delta y}{\Delta x}$
If $\Delta y = O$ and $\Delta x = A$, then the slope is the Tangent.\\
Therefore, $m = \tan{\theta}$\\
The slope of a 2D line is simply the tangent of the smaller positive angle between the line and the x axis.\\
We can therefore rewrite the equation $y = mx + b$ into $y = \tan{\theta}x + b$\\
At $\frac{\pi}{2}, \frac{-\pi}{2}$, the slope is undefined (Vertical Line)\\

Likewise, in $\mathbb{R}^3$, a line, $L$,  is determined by a point $P_0 = (X_0, Y_0, Z_0)$ and a direction $v$.\\
We can position our vector so that it's tail lands on $P_0$.\\
Select randomly a point $P = (x, y, z)$, where $x, y, z$ are unknown, such that P belongs to the line $L$.\\
Can we characterize what $x, y, z$ are?

We can draw a vector $\overrightarrow{OP_0}$, such that the tail lands on the origin, and the head lies on $P_0$\\
$\overrightarrow{P_0P}$ is a parallel vector $v$\\
In other words, there exists some unknown scalar $t \in \mathbb{R}$ (Parameter), such that $\overrightarrow{P_0P}=tv$\\
By the geometrix properties of vector addition, $\overrightarrow{OP} = \overrightarrow{OP_0} + tv$\\
Vector $\overrightarrow{OP} = \langle x,y,z \rangle = \langle x_0, y_0, z_0\rangle + tv \equiv \langle x_0+at, y_0+bt, z_0+ct\rangle$\\
Thus, in summary,\\
$x = x_0 + at$,\\
$y = y_0 + at$\\
$z = z_0 + at$\\
Parametric Equations of Line $L$\\

\begin{equation*}
    r(t) = \langle x_0+at, y_0+bt, z_0+ct\rangle
\end{equation*}
This is called the ``Vector Equation of $L$''

\end{document}